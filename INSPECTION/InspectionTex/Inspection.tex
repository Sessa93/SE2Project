% !TEX TS-program = pdflatex
% !TEX encoding = UTF-8 Unicode

% This is a simple template for a LaTeX document using the "article" class.
% See "book", "report", "letter" for other types of document.

\documentclass[11pt,titlepage]{article} % use larger type; default would be 10pt

\usepackage[utf8]{inputenc} % set input encoding (not needed with XeLaTeX)

%%% Examples of Article customizations
% These packages are optional, depending whether you want the features they provide.
% See the LaTeX Companion or other references for full information.

%%% PAGE DIMENSIONS
\usepackage{geometry} % to change the page dimensions
\geometry{a4paper} % or letterpaper (US) or a5paper or....
% \geometry{margin=2in} % for example, change the margins to 2 inches all round
% \geometry{landscape} % set up the page for landscape
%   read geometry.pdf for detailed page layout information

\usepackage{graphicx} % support the \includegraphics command and options
\usepackage{titlepic}

% \usepackage[parfill]{parskip} % Activate to begin paragraphs with an empty line rather than an indent

%%% PACKAGES
\usepackage{booktabs} % for much better looking tables
\usepackage{array} % for better arrays (eg matrices) in maths
\usepackage{paralist} % very flexible & customisable lists (eg. enumerate/itemize, etc.)
\usepackage{verbatim} % adds environment for commenting out blocks of text & for better verbatim
\usepackage{subfig} % make it possible to include more than one captioned figure/table in a single float
% These packages are all incorporated in the memoir class to one degree or another...

%%% HEADERS & FOOTERS
\usepackage{fancyhdr} % This should be set AFTER setting up the page geometry
\pagestyle{plain} % options: empty , plain , fancy
\renewcommand{\headrulewidth}{0pt} % customise the layout...
\lhead{}\chead{}\rhead{}
\lfoot{}\cfoot{\thepage}\rfoot{}

%%% SECTION TITLE APPEARANCE
\usepackage{sectsty}
\allsectionsfont{\sffamily\mdseries\upshape} % (See the fntguide.pdf for font help)
% (This matches ConTeXt defaults)

%%% ToC (table of contents) APPEARANCE
\usepackage[nottoc,notlof,notlot]{tocbibind} % Put the bibliography in the ToC
\usepackage[titles,subfigure]{tocloft} % Alter the style of the Table of Contents
\renewcommand{\cftsecfont}{\rmfamily\mdseries\upshape}
\renewcommand{\cftsecpagefont}{\rmfamily\mdseries\upshape} % No bold!

\newenvironment{changemargin}[3]{%
\begin{list}{}{%
\setlength{\topsep}{0pt}%
\setlength{\headsep}{#3}%
\setlength{\leftmargin}{#1}%
\setlength{\rightmargin}{#2}%
\setlength{\listparindent}{\parindent}%
\setlength{\itemindent}{\parindent}%
\setlength{\parsep}{\parskip}%
}%
\item[]}{\end{list}}

%Table Formatting
\usepackage{tabularx,hhline}
\usepackage{pbox}

\usepackage{listings}
\usepackage{color}

\definecolor{myblue}{rgb}{0.031,0.56,0.741}
\definecolor{mygray}{rgb}{0.5,0.5,0.5}
\definecolor{mymauve}{rgb}{0.58,0,0.82}
\definecolor{lightgray}{rgb}{0.988,0.988,0.988}

\lstset{
   backgroundcolor=\color{lightgray},
   extendedchars=true,
   basicstyle=\footnotesize\ttfamily,
   showstringspaces=false,
   showspaces=false,
   numbers=left,
   numberstyle=\footnotesize,
   numbersep=7pt,
   tabsize=2,
   breaklines=true,
   showtabs=false,
   captionpos=b,
   commentstyle=\color{myblue},
   keywordstyle=\color{blue},
   stringstyle=\color{mymauve},
}

%%% END Article customizations

%%% The "real" document content comes below...

\titlepic{\includegraphics[scale=0.60]{polimi_logo.jpg}}
\title{Inspection Document \\ \vspace{1cm} \large{Version 1.0}} 
\author{Giorgio Pea(Mat. 853872), Andrea Sessa(Mat. 850082)}
\date{5/1/2016} 

\begin{document}

\maketitle

\newpage

\tableofcontents

\newpage

\section{Introduction}

\section{Classes}

Included in this section the two java classes subjected to the analysis.\newline

\noindent File: /appserver/web/web-core/src/main/java/org/apache/catalina/ssi/\textbf{SSIServlet.java}\newline\newline
Methods under inspection:
  \begin{itemize}
    \item \textbf{\textit{init()}}
    \item \textbf{\textit{requestHandler(HttpServletRequest req, HttpServletResponse res)}}
    \item \textbf{\textit{processSSI( HttpServletRequest req , HttpServletResponse res , URL resource )}}
  \end{itemize}

\noindent File: /appserver/web/web-core/src/main/java/org/apache/catalina/ssi/\textbf{SSIMediator.java}\newline\newline
Methods under inspection:
  \begin{itemize}
    \item \textbf{\textit{substituteVariables(String val)}}
  \end{itemize}
  
\newpage


\section{Functional Role}
  In this section are included some information about the functioning of tha analyzed classes and methods.\newline

\subsection{SSIServlet.java}

  From the Javadoc:
  
  \lstinputlisting[language=Java, firstnumber=78, firstline=78, lastline=81]{SSIServlet.java}

  \noindent This class represents a Java EE servlet used to process requests that include some SSI instruction.\newline
  SSI(Server Side Include) that is a simple interpreted server-side scripting language.
  The most frequent use of SSI is to include the contents of one or more files into a web page on a web server.

  \begin{itemize}
    \item \textbf{\textit{Init()}}
      \newline From the javadoc of the method:
      \lstinputlisting[language=Java, firstnumber=104, firstline=104, lastline=109]{SSIServlet.java}
      It is clear that the method take care of initialize the Java EE servlet by retrieving the configuration
      parameters(through the methods provided by the superclass: `GenericServlet`)\newline
      Follows an explanation of the initialization parameters(obtained by analyzing the comments provided with the method):
      \begin{itemize}
       \item \textbf{debug}:
	Specifies the debug level of the servlet, if 0 no debug message are logged
       \item \textbf{isVirtualWebappRelative}:
	Specifies if the paths can be webapp relative
       \item \textbf{expires}:
	Specifies the expiration time of this servlet(in seconds)
       \item \textbf{buffered}:
	Specifies if the output response of the servlet should be buffered first or not(see processSSI() for more details)
       \item \textbf{inputEncoding}:
	Specifies the encoding of the HttpServletRequest
       \item \textbf{outputEncoding}:
	Specifies the encoding of the HttpServletResponse
      \end{itemize}

      

    \item \textbf{\textit{requestHandler()}}
      \newline From the inspection of the code this function is only called when the servlet receives a HTTP Get or Post request.
      The javadoc for the method, included in the code, states:
      
      \lstinputlisting[language=Java, firstnumber=173, firstline=173, lastline=180]{SSIServlet.java}
      
      Hence the method accepts as parameters a HttpServletRequest,the incoming request, and a HttpServletResponse that is a reference
      to the response.\newline
      
      Now the objective of the method is to retrieve the correct resource from the serverlet context.
      If the debug level is greater than zero then log a message into the logger for debug purposes.
      \lstinputlisting[language=Java, firstnumber=183, firstline=183, lastline=188]{SSIServlet.java}

      The comment is very clear: it checks if the resource is either in the `WEB-INF` or `META-INF` subdirectories;
      if so the function return with an error code.
      \lstinputlisting[language=Java, firstnumber=189, firstline=189, lastline=196]{SSIServlet.java}
      
      Here the function tries to retrieve the URL to the resource; it also performs an existence check on the resource,
      if the resource doesn't exist the function return an error.
      \lstinputlisting[language=Java, firstnumber=197, firstline=197, lastline=202]{SSIServlet.java}
      
      In the final part, the function starts to initialize the header of the HttpServletResponse by setting:
      the mime type, the encoding of the output text and the expiration time for the response(in seconds, see init()).\newline
      Finally the processSSI() function is invoked passing as parameters the original request, the reference to the response and the resource.
      \lstinputlisting[language=Java, firstnumber=203, firstline=203, lastline=213]{SSIServlet.java}
      \newpage
      
    \item \textbf{\textit{processSSI()}}
      The method is totally uncommented, but thanks to a meaningful choice of the variables names it is quite easy to understand
      the role of the method within the class.
      The objectives of processSSI() are:
	\begin{itemize}
	  \item Parse the SSI code contained in the resource(passed as parameter) via the SSIProcessor class
	  \item Write the output of the parsing phase in the response(passed as parameter)
	\end{itemize}
      The lines:
	\lstinputlisting[language=Java, firstnumber=233, firstline=233, lastline=245]{SSIServlet.java}
      take care of the initialization of the stream used to parse the resource(`InputStream`) and retrieves the encoding
      of the data contained in the resource
      
      The lines:
      \lstinputlisting[language=Java, firstnumber=222, firstline=222, lastline=222]{SSIServlet.java}
      \lstinputlisting[language=Java, firstnumber=247, firstline=247, lastline=248]{SSIServlet.java}
      parse the SSI code contained in the resource.
      
      The two blocks:
      \lstinputlisting[language=Java, firstnumber=226, firstline=226, lastline=231]{SSIServlet.java}
      \lstinputlisting[language=Java, firstnumber=252, firstline=252, lastline=256]{SSIServlet.java}
      Initialize the streams to write the output of the parsing phase also taking into account the necessity
      to buffer the output stream(buffered boolean variable).\newline
      If buffered is true the output of the parser is written first to a buffer(StringWriter) and then, only in a separate moment
      the output is written to the actual destination.Otherwise if buffered is equal to false then SSIProcessor(the actual SSI parser)
      writes the parsed information directly to the output stream.

  \end{itemize}

\subsection{SSIMediator.java}
  From the Javadoc of the class:
  
  \lstinputlisting[language=Java, firstnumber=75, firstline=75, lastline=78]{SSIMediator.java}
  
  The class is inserted into the context of SSI processing, in particular this class take care of
  how many different implementations of the SSI instructions can communicate and exchange data with each other.\newline
  Follows a detailed description of the assigned methods:
  \begin{itemize}
   \item \textbf{\textit{substituteVariables()}}
    \lstinputlisting[language=Java, firstnumber=246, firstline=246, lastline=249]{SSIMediator.java}
    The method accepts as parameter a string and returns a new string to which a variables substitution process has been applied.
    \lstinputlisting[language=Java, firstnumber=251, firstline=251, lastline=253]{SSIMediator.java}
    It checks if the string contains `\$` or `\&`, if not there is nothing to substitute so the original string is simply returned.
    Otherwise: 
    \lstinputlisting[language=Java, firstnumber=253, firstline=253, lastline=259]{SSIMediator.java}
    It's easy to understand(from the comments and javadoc) that above snippet of code substitute each occurrence of HTML special codes with the real character. \newline
    \lstinputlisting[language=Java, firstnumber=261, firstline=261, lastline=274]{SSIMediator.java}
    This part of the code takes care of substuting `\&\#n` with `n` where `n` is an integer number.
    See the javadoc of StringBuilder(Java SE 7 class) for a detailed explanation of the methods.\newline
    The remaining code processes variables and substitutes their current value.\newline
    Variables are always in the form `\$ varName` and could possibly be wrapped, ie. `\$\{varName\}`. 
    This information has been collected by an direct analysis of the code and by means of the few comments inserted. 
    The actual value of the variables found in the string are retrieved by means of the `getVariablesValue()` function(also defined in SSIMediator.java).\newline
    Find the first `\$`, eventually escaped.
    \lstinputlisting[language=Java, firstnumber=277, firstline=277, lastline=290]{SSIMediator.java}
    The following code identifies the portion string to substitute [nameStart, nameEnd] and the name of the variable [start, end].
    Also the functions consider the possibility that the variable could be wrapped so it processes the presence of `\{` and `\}` that
    are wrapping the variable name.
    \lstinputlisting[language=Java, firstnumber=291, firstline=291, lastline=307]{SSIMediator.java}
    Finally the variable name has been identified in the original string [start, end] and the `getVariablesValue()` method is called
    to retrieve the value of the variable. The value is then substituted and the function seeks for the presence of other variables.
    If no more variables are found, the function returns the processed string.
    \lstinputlisting[language=Java, firstnumber=308, firstline=308, lastline=318]{SSIMediator.java}


  \end{itemize}


\section{Issues}
In this section is included a list of problems found during the inspection of the assigned code.
\subsection{SSIServlet.java}
  \textbf{General Considerations} \hfill \\
  In general the class lacks of documentation: comments and javadoc are not complete and where inserted are sometimes meaningless and very short.
  \begin{itemize}
   \item \textbf{\textit{init()}}
   \begin{enumerate}
    \item Checklist[11]: All the if statements present in the body of this method do not use curly braces
    \item Checklist[23]: The javadoc written for this method is not sufficient to explain its role and its behavior in the context of the SSIServlet class
    \item Checklist[40]: All the comparisons present in the body of this method use improper operators, in fact the elements in comparison are always objects(strings in particular)
    \item Checklist[18]: None of the instructions present in the body of this method is commented. This may be correct if all these instructions are self explicative, but at least the last if statement needs comments to explain what it tries to achieve
    \item Checklist[14]: Lines
      \lstinputlisting[language=Java, firstnumber=113, firstline=113, lastline=113]{SSIServlet.java}
      \lstinputlisting[language=Java, firstnumber=116, firstline=116, lastline=116]{SSIServlet.java}
      \lstinputlisting[language=Java, firstnumber=119, firstline=119, lastline=119]{SSIServlet.java}
      \lstinputlisting[language=Java, firstnumber=121, firstline=121, lastline=121]{SSIServlet.java}
      \lstinputlisting[language=Java, firstnumber=126, firstline=126, lastline=126]{SSIServlet.java}
      exceed the length of 80 characters
    \item Checklist[8,9]: The indentation of lines is made using tabs and not spaces
    \item Checklist[52,53]: In the line
      \lstinputlisting[language=Java, firstnumber=119, firstline=119, lastline=119]{SSIServlet.java}
      the Long.valueOf method throws a NumberFormatException which is not managed and must be imported
      \lstinputlisting[language=Java, firstnumber=113, firstline=113, lastline=113]{SSIServlet.java}
      The Integer.parseInt method throws a NumberFormatException which is not managed and must be imported.
   \end{enumerate}

   
   \item \textbf{\textit{requestHandler()}}
    \begin{enumerate}
     \item Checklist[8,9]: All indentations in the class are made by means of tabs
     \item Checklist[11]: The conditional block
      \lstinputlisting[language=Java, firstnumber=185, firstline=185, lastline=188]{SSIServlet.java}
      uses no enclosing braces
     \item Checklist[18]: No comments from line 210 to the end of the function
     \item Checklist[29,33]: The declarations of variables in lines
      \lstinputlisting[language=Java, firstnumber=197, firstline=197, lastline=197]{SSIServlet.java}
      \lstinputlisting[language=Java, firstnumber=203, firstline=203, lastline=203]{SSIServlet.java}
      should be placed at the start of the function block
     \item Checklist[40]: The lines
      \lstinputlisting[language=Java, firstnumber=191, firstline=191, lastline=191]{SSIServlet.java}
      \lstinputlisting[language=Java, firstnumber=198, firstline=198, lastline=198]{SSIServlet.java}
      \lstinputlisting[language=Java, firstnumber=204, firstline=204, lastline=204]{SSIServlet.java}
      \lstinputlisting[language=Java, firstnumber=208, firstline=208, lastline=208]{SSIServlet.java}
      uses for comparation `==` instead of `equals()`
     \item Checklist[52,53]: The line 
      \lstinputlisting[language=Java, firstnumber=197, firstline=197, lastline=197]{SSIServlet.java}
      may throws a `MalformedURLException`, neither actions are taken to manage the exception nor the exception is explicitly re-thrown
     \item Checklist[15]: Wrong line breaking in
      \lstinputlisting[language=Java, firstnumber=191, firstline=191, lastline=192]{SSIServlet.java}

    \end{enumerate}
    
   \item \textbf{\textit{processSSI()}}
    \begin{itemize}
     \item Checklist[23]: No javadoc has been written for this method
     \item Checklist[18]: None of the instructions present in the body of this method is commented. This may be
      correct if all these instructions are self explicative, but most of the instructions
      present in this method are not self explicative
     \item Checklist[40]: All the comparisons present in the body of this method use improper
      operators(== or !== instead of equals or !..equals), in fact the elements in comparison
      are always objects(strings in particular)
     \item Checklist[29,33]: In these lines
      \lstinputlisting[language=Java, firstnumber=233, firstline=233, lastline=235]{SSIServlet.java}
      \lstinputlisting[language=Java, firstnumber=239, firstline=239, lastline=239]{SSIServlet.java}
      local variables are defined and assigned to a value. Since these assigments and definitions
      do not depend from the result of previous instructions, they must be put in the top
      of the body of the method
     \item Checklist[1]:
      \lstinputlisting[language=Java, firstnumber=239, firstline=239, lastline=239]{SSIServlet.java}
      In this line a local variable of the type "InputStreamReader" is defined. The name of
      this variable is "isr" which does not convey any immediate meaning about the role and the
      use of this variable
     \item Checklist[52,53]:
      \lstinputlisting[language=Java, firstnumber=233, firstline=233, lastline=234]{SSIServlet.java}
      The method openConnection throws an IOException that is not managed\newline
      The method getInputStream throws an IOException that is not managed\newline\newline
      \lstinputlisting[language=Java, firstnumber=230, firstline=230, lastline=230]{SSIServlet.java}
      \lstinputlisting[language=Java, firstnumber=255, firstline=255, lastline=255]{SSIServlet.java}
      The getWriter method on the HttpServletResponse object throws a IOException, a IllegalStateException, UnsupportedEncodingException
      which are neither managed nor imported (IllegalStateException)
     \item Checklist[58]:
      \lstinputlisting[language=Java, firstnumber=247, firstline=247, lastline=248]{SSIServlet.java}
      The method close should be invoked on the bufferedReader variable and on the
      isr variable, since these variables are not used anymore in the rest of the method
      and they represent streams of bytes readers
    \end{itemize}

  \end{itemize}


\subsection{SSIMediator.java}
  \textbf{General Considerations} \hfill \\
  In general the class lacks of documentation: the javadoc is not complete and many instructions blocks are left with no comments at all.
  \begin{itemize}
   \item \textbf{\textit{substituteVariables()}}
   \item Checklist[8,9]: Tabs are used for identation for all the function
   \item Checklist[23]: The Javadoc provided for the function is not complete 
   \item Checklist[11]: No enclosing braces in the following if statements:
    \lstinputlisting[language=Java, firstnumber=253, firstline=253, lastline=253]{SSIMediator.java}
    \lstinputlisting[language=Java, firstnumber=303, firstline=303, lastline=303]{SSIMediator.java}
    \lstinputlisting[language=Java, firstnumber=307, firstline=307, lastline=307]{SSIMediator.java}
    \lstinputlisting[language=Java, firstnumber=311, firstline=311, lastline=311]{SSIMediator.java}
    
   \item Checklist[1]: The parameter of the function is named `val` which is not meaningfull to understand its role in the function execution
   
   \item Checklist[15]: Wrong line breking in the line
    \lstinputlisting[language=Java, firstnumber=266, firstline=266, lastline=267]{SSIMediator.java}

   \item Checklist[52,53]: No action are taken in case one of the following lines throws a NullPointerException:
    \lstinputlisting[language=Java, firstnumber=262, firstline=262, lastline=262]{SSIMediator.java}
    \lstinputlisting[language=Java, firstnumber=264, firstline=264, lastline=264]{SSIMediator.java}
    \lstinputlisting[language=Java, firstnumber=270, firstline=270, lastline=270]{SSIMediator.java}
    
   %\item Checklist[52,53]: No action are taken in case one of the following lines throws IndexOutOfBoundsException:
    %\lstinputlisting[language=Java, firstnumber=266, firstline=266, lastline=267]{SSIMediator.java}
    %\lstinputlisting[language=Java, firstnumber=268, firstline=268, lastline=268]{SSIMediator.java}
    %\lstinputlisting[language=Java, firstnumber=269, firstline=269, lastline=269]{SSIMediator.java}
    %\lstinputlisting[language=Java, firstnumber=279, firstline=279, lastline=279]{SSIMediator.java}
    %\lstinputlisting[language=Java, firstnumber=286, firstline=286, lastline=286]{SSIMediator.java}
    %\lstinputlisting[language=Java, firstnumber=287, firstline=287, lastline=287]{SSIMediator.java}
    %\lstinputlisting[language=Java, firstnumber=297, firstline=297, lastline=297]{SSIMediator.java}
    %\lstinputlisting[language=Java, firstnumber=313, firstline=313, lastline=313]{SSIMediator.java}
    
   \item Checklist[40]: The line
    \lstinputlisting[language=Java, firstnumber=311, firstline=311, lastline=311]{SSIMediator.java}
    uses for comparation `==` instead of `equals()`
   \item Checklist[33]: The variables in lines
    \lstinputlisting[language=Java, firstnumber=309, firstline=309, lastline=310]{SSIMediator.java}
    should be placed at the top of the function

  \end{itemize}
\newpage
\section{Additional Issues}
  In this section are inclued additional problems and issues not present in the checklist:
  \subsection{SSIServlet.java}
    \begin{itemize}
      \item \textbf{\textit{init()}}
	\begin{itemize}
	 \item In these lines:
	    \lstinputlisting[language=Java, firstnumber=115, firstline=115, lastline=116]{SSIServlet.java}
	    \lstinputlisting[language=Java, firstnumber=121, firstline=121, lastline=121]{SSIServlet.java}
	    \lstinputlisting[language=Java, firstnumber=123, firstline=123, lastline=123]{SSIServlet.java}
	    we have the assignment of properties of the class, and this assignment does not
	    depend from the result of previous instructions. Given that, these instructions should be
	    put in the top of the body of the method
	 \item In the body of this method continuos calls to the methods of the object returned by the
	    getServletConfig() method are performed.
	    This is inefficient since the above mentioned object can be stored in a local variable
	    and so made accessible without method calls
	 \item Methods and properties of the superclass of a class must be referenced by that class using
	    the "super." prefix. This should be done for reasons of clarity and readability, so that
	    the developer can immediately distinguish the manipulation of methods and properties of the superclass of the current class
	    from the manipulation of those which belong to the current class.\newline
	    This behavior is not followed in this method(all lines)

	\end{itemize}

      \item \textbf{\textit{requestHandler()}}
	  \begin{itemize}
	      \item Methods and properties of the superclass of a class must be referenced by that class using
		the "super" prefix. This should be done for reasons of clarity and readability, so that
		the developer can immediately distinguish the manipulation of methods and properties of the superclass of the current class
		from the manipulation of those which belong to the current class.\newline
		This behavior is not followed in lines:
		\lstinputlisting[language=Java, firstnumber=183, firstline=183, lastline=183]{SSIServlet.java}
		And in all lines that present the invocation of `log()`

	      \item Methods and properties of the current class must be referenced within the class using the "this." prefix.
		This should be done for reasons of clarity and readability, so that the developer can immediately distinguish
		the manipulation of methods and properties of the current class from the manipulation of those which belong
		to the superclass of the current class.\newline This is also useful for distinguish the manipulation of properties of
		the current class and local variables.
		This behavior is not followed in lines:
		\lstinputlisting[language=Java, firstnumber=185, firstline=185, lastline=188]{SSIServlet.java}
		\lstinputlisting[language=Java, firstnumber=207, firstline=207, lastline=208]{SSIServlet.java}
	      
	      \item In lines
		\lstinputlisting[language=Java, firstnumber=186, firstline=186, lastline=188]{SSIServlet.java}
		the ternary operator ? is used. The expression is syntactically valid but the use of ? makes it counter intuitive and less readable.\newline 
		It is preferable to use a classic if-else block.
	   \end{itemize}
	    
      \item \textbf{\textit{processSSI()}}
	    \begin{itemize}
	     \item Methods and properties of the superclass of a class must be referenced by that class using
		the "super" prefix. This should be done for reasons of clarity and readability, so that
		the developer can immediately distinguish the manipulation of methods and properties of the superclass of the current class
		from the manipulation of those which belong to the current class.\newline
		This behavior is not followed in this method(all lines).

	     \item Methods and properties of the current class must be referenced within the class using the "this." prefix.
		This should be done for reasons of clarity and readability, so that the developer can immediately distinguish
		the manipulation of methods and properties of the current class from the manipulation of those which belong
		to the superclass of the current class.\newline This is also useful for distinguish the manipulation of properties of
		the current class and local variables.
		This behavior is not followed in lines:
		\lstinputlisting[language=Java, firstnumber=220, firstline=220, lastline=223]{SSIServlet.java}
		\lstinputlisting[language=Java, firstnumber=226, firstline=226, lastline=226]{SSIServlet.java}
		\lstinputlisting[language=Java, firstnumber=252, firstline=252, lastline=252]{SSIServlet.java}
		
	     \item In the following two lines of code:
		\lstinputlisting[language=Java, firstnumber=224, firstline=224, lastline=225]{SSIServlet.java}
		each statement declares a variable and then assigns to it a `null` value. In general assign a `null` value to a fresh declared
		variable is an useless operation, indeed this is the default behavior of Java.
		
	     \item In the following block of code:
		\lstinputlisting[language=Java, firstnumber=236, firstline=236, lastline=244]{SSIServlet.java}
		the first if statement is redundant, it should be deleted and its contents copied into the body of the second if statement.
	     \item In the following line:
		\lstinputlisting[language=Java, firstnumber=233, firstline=233, lastline=234]{SSIServlet.java}
		  The getInputStream and getContentEncoding methods cannot be called on a "URLConnection" object before opening an actual connection
		  to the resource referred by object itself (to solve this problem the "connect" method must be called on the
		  "URLConnection" object before calling getInputStream, see the javadoc of the URL class)
	    \end{itemize}
   \end{itemize}
  \subsection{SSIMediator.java}
    No particular additional issues has been found in the `substituteVariables()` method.

\newpage
\section{Appendix}
\subsection{References}
   \begin{itemize}
    \item javaCheckList.pdf: Contains the check list used to inspect the code present in this document.
   \end{itemize}
\subsection{Tools Used}
  \begin{itemize}
   \item Atom/ \LaTeX: To redact this document
   \item Eclipse: To simulate the behavior of the assigned code
  \end{itemize}
\subsection{Hours of Work}
  \begin{itemize}
   \item Andrea Sessa: xxx hours
   \item Giorgio Pea: xxx hours
  \end{itemize}

\end{document}
