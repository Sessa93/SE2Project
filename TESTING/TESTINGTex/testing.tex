% !TEX TS-program = pdflatex
% !TEX encoding = UTF-8 Unicode

% This is a simple template for a LaTeX document using the "article" class.
% See "book", "report", "letter" for other types of document.

\documentclass[11pt,titlepage]{article} % use larger type; default would be 10pt

\usepackage[utf8]{inputenc} % set input encoding (not needed with XeLaTeX)

%%% Examples of Article customizations
% These packages are optional, depending whether you want the features they provide.
% See the LaTeX Companion or other references for full information.

%%% PAGE DIMENSIONS
\usepackage{geometry} % to change the page dimensions
\geometry{a4paper} % or letterpaper (US) or a5paper or....
% \geometry{margin=2in} % for example, change the margins to 2 inches all round
% \geometry{landscape} % set up the page for landscape
%   read geometry.pdf for detailed page layout information

\usepackage{graphicx} % support the \includegraphics command and options
\usepackage{titlepic}

% \usepackage[parfill]{parskip} % Activate to begin paragraphs with an empty line rather than an indent

%%% PACKAGES
\usepackage{booktabs} % for much better looking tables
\usepackage{array} % for better arrays (eg matrices) in maths
\usepackage{paralist} % very flexible & customisable lists (eg. enumerate/itemize, etc.)
\usepackage{verbatim} % adds environment for commenting out blocks of text & for better verbatim
\usepackage{subfig} % make it possible to include more than one captioned figure/table in a single float
% These packages are all incorporated in the memoir class to one degree or another...

%%% HEADERS & FOOTERS
\usepackage{fancyhdr} % This should be set AFTER setting up the page geometry
\pagestyle{fancy} % options: empty , plain , fancy
\renewcommand{\headrulewidth}{0pt} % customise the layout...
\lhead{}\chead{}\rhead{}
\lfoot{}\cfoot{\thepage}\rfoot{}

%%% SECTION TITLE APPEARANCE
\usepackage{sectsty}
\allsectionsfont{\sffamily\mdseries\upshape} % (See the fntguide.pdf for font help)
% (This matches ConTeXt defaults)

%%% ToC (table of contents) APPEARANCE
\usepackage[nottoc,notlof,notlot]{tocbibind} % Put the bibliography in the ToC
\usepackage[titles,subfigure]{tocloft} % Alter the style of the Table of Contents
\renewcommand{\cftsecfont}{\rmfamily\mdseries\upshape}
\renewcommand{\cftsecpagefont}{\rmfamily\mdseries\upshape} % No bold!

\newenvironment{changemargin}[3]{%
\begin{list}{}{%
\setlength{\topsep}{0pt}%
\setlength{\headsep}{#3}%
\setlength{\leftmargin}{#1}%
\setlength{\rightmargin}{#2}%
\setlength{\listparindent}{\parindent}%
\setlength{\itemindent}{\parindent}%
\setlength{\parsep}{\parskip}%
}%
\item[]}{\end{list}}

%Table Formatting
\usepackage{tabularx,hhline}
\usepackage{pbox}

%%% END Article customizations

%%% The "real" document content comes below...

\titlepic{\includegraphics[scale=0.60]{polimi_logo.jpg}}
\title{Integration Test Plan Document \\ \vspace{1cm} \large{Version 1.0}}
\author{Giorgio Pea(Mat. 853872), Andrea Sessa(Mat. 850082)}
\date{21/1/2016}

\begin{document}

\maketitle

\newpage

\tableofcontents

\newpage

\section{Introduction}

\subsection{Revision History}
	\begin{table}[h]
	  \centering
	  \begin{tabular}{lll}
	  \textbf{Date} & \textbf{Description} & \textbf{Authors} \\
	    21/01/2016 & Delivers of version 1.0 & Pea, Sessa
	  \end{tabular}
	\end{table}

\subsection{Purpose}
  The purpose of the integration test plan is to describe the necessary tests to verify that all the components(see \textbf{DD}, reference section)
  of MyTaxiServive are properly assembled.
  Integration testing ensures that the unit-tested components interact correctly. \newline
  
\subsection{Scope}
    The aim of this project is to develop MyTaxiService, a web/mobile application that
    makes easier and quicker taking taxies within the city’s borders. Thanks to MyTaxiService,
    anyone can request or book a taxi and get realtime information about how long
    it will take to be picked up or about the taxi’s current position and identification code.
    In addition to that, MyTaxiService provides an efficient way to allocate taxies by dividing
    the city in zones and using a queue based allocation system, in order to reduce the
    average waiting time and city’s traffic.\newline

\subsection{Terms Definition}
  \subsubsection{Glossary}
    \begin{itemize}
      \item \textbf{Mtaxi:} A taxi that joined MyTaxiService
      \item \textbf{User:} Refers to either a logged registered user or a generic user(see RASD)
      \item \textbf{MyTaxiService(B):} see RASD
      \item \textbf{Administrator:} see RASD
      \item \textbf{Mtaxi bad beahvior}: see RASD
    \end{itemize}
   
  \subsubsection{Acronyms}
    \begin{itemize}
      \item \textbf{DD:} Design Document
      \item \textbf{MVC:} Model View Controller
      \item \textbf{FIFO:} First In First Out
      \item \textbf{API:} Application Programming Interface
      \item \textbf{GUI:} Graphic User Interface
      \item \textbf{GPS:} Global Positioning System
      \item \textbf{AWT:} Approximate waiting time to be picked up
     \end{itemize}

\subsection{Reference Documents}
  \begin{itemize}
   \item \textbf{[RASD]}Requirements and analisys specification document(RASD)
   \item \textbf{[DD]} Design Document(DD)
  \end{itemize}

\newpage

\section{Integration Strategy}

\subsection{Overview}
  In the first part of this section the components to be tested are mentioned,
  moreover the testing strategy is chosen and the integration testing order defined.\newline
  A more detailed specification for each test case is given in the third chapter.

\subsection{Entry Criteria}
  The main entry conditions for this phase is that each of the system components low level functions
  has been previously subjected to a unit test process. 

\subsection{Elements to be integrated}
  For a detailed description of each components function and interaction refer to the \textbf{DD} \newline
  Two main subsystem can be individuated in the general architecture of MyTaxiService:
  \begin{enumerate}
    \item \textbf{Client side Component}
	  \begin{itemize}
	    \item MyTaxiServiceAppUser GUI
	    \item MyTaxiServiceMYT GUI
	    \item MyTaxiServiceWebUser GUI
	    \item MyTaxiServiceWebAdmin GUI
	    
	    \item MyTaxiServiceAppUser Communication Manager
	    \item MyTaxiServiceMYT Communication Manager
	    \item MyTaxiServiceWebUser Communication Manager
	    \item MyTaxiServiceWebAdmin Communication Manager
	  \end{itemize}
    
    \item \textbf{Server side Component}
	  \begin{itemize}
	    \item Dispatcher
	    \item Request Manager
	    \item External Services Manager
	    \item Location Manager
	    \item Queue Manager
	    \item Request Receiver
	    \item Data Manager
	  \end{itemize}
  \end{enumerate}


\subsection{Integration Testing Strategy}
  The strategy chosen for the integration of the components of the application is a grouped Bottom-Up integration: 
  for each subsystem S of MyTaxiService, its components are orchestrated together following a bottom-up approach, 
  when this procedure is finished a general integration of the subsystems of the application with each other is performed.\newline
  This integration testing strategy has been selected because it mimics very well the structure and the modularity of the software system.
  The bottom-up approach consists in integrating first those components that are at the low level of the component's dependency tree
  (see DD's component diagram).
  
\subsection{Sequence of Component/Function Integration} 
  \subsubsection{Software Integration Sequence}
    \includegraphics[scale=0.3]{Integration-1.png}
  \subsubsection{Subsystem Integration Sequence}

\section{Individual Step and Test Description}

\section{Tools and Test Equipment Required}

\section{Program Stub and Test Data Required}


\end{document}
