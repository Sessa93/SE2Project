% !TEX TS-program = pdflatex
% !TEX encoding = UTF-8 Unicode

% This is a simple template for a LaTeX document using the "article" class.
% See "book", "report", "letter" for other types of document.

\documentclass[11pt]{article} % use larger type; default would be 10pt

\usepackage[utf8]{inputenc} % set input encoding (not needed with XeLaTeX)

%%% Examples of Article customizations
% These packages are optional, depending whether you want the features they provide.
% See the LaTeX Companion or other references for full information.

%%% PAGE DIMENSIONS
\usepackage{geometry} % to change the page dimensions
\geometry{a4paper} % or letterpaper (US) or a5paper or....
% \geometry{margin=2in} % for example, change the margins to 2 inches all round
% \geometry{landscape} % set up the page for landscape
%   read geometry.pdf for detailed page layout information

\usepackage{graphicx} % support the \includegraphics command and options

% \usepackage[parfill]{parskip} % Activate to begin paragraphs with an empty line rather than an indent

%%% PACKAGES
\usepackage{booktabs} % for much better looking tables
\usepackage{array} % for better arrays (eg matrices) in maths
\usepackage{paralist} % very flexible & customisable lists (eg. enumerate/itemize, etc.)
\usepackage{verbatim} % adds environment for commenting out blocks of text & for better verbatim
\usepackage{subfig} % make it possible to include more than one captioned figure/table in a single float
% These packages are all incorporated in the memoir class to one degree or another...

%%% HEADERS & FOOTERS
\usepackage{fancyhdr} % This should be set AFTER setting up the page geometry
\pagestyle{fancy} % options: empty , plain , fancy
\renewcommand{\headrulewidth}{0pt} % customise the layout...
\lhead{}\chead{}\rhead{}
\lfoot{}\cfoot{\thepage}\rfoot{}

%%% SECTION TITLE APPEARANCE
\usepackage{sectsty}
\allsectionsfont{\sffamily\mdseries\upshape} % (See the fntguide.pdf for font help)
% (This matches ConTeXt defaults)

%%% ToC (table of contents) APPEARANCE
\usepackage[nottoc,notlof,notlot]{tocbibind} % Put the bibliography in the ToC
\usepackage[titles,subfigure]{tocloft} % Alter the style of the Table of Contents
\renewcommand{\cftsecfont}{\rmfamily\mdseries\upshape}
\renewcommand{\cftsecpagefont}{\rmfamily\mdseries\upshape} % No bold!

\renewcommand{\labelitemi}{$\bullet$}
\renewcommand{\labelitemii}{$\cdot$}
\renewcommand{\labelitemiii}{$\diamond$}
\renewcommand{\labelitemiv}{$\ast$}

%%% END Article customizations

%%% The "real" document content comes below...

\title{RASD}
\author{Giorgio Pea, Andrea Sessa}
%\date{} % Activate to display a given date or no date (if empty),
         % otherwise the current date is printed

\begin{document}
\maketitle
\newpage

\tableofcontents

\newpage

\section{Introduction}
  \subsection{Purpose}
    This document represent the Requirement Analysis and Specification Document
    (RASD). The main goal of this document is to completely describe the system
    in terms of functional and non-functional requirements, to show the constraints and the limit
    of the software and simulate the typical use cases that will occur after the
    development. This document is intended to all developer and programmer who
    have to implement the requirements, to system analyst who want to integrate
    other system with this one, and could be used as a contractual basis between
    the customer and the developer.

  \subsection{Scope of the project}
    The aim of this project is to develop an application myTaxiService, a web/mobile applications that makes easier and quicker taking taxies.
    Thanks to MyTaxiService, anyone can request or book a taxi and get realtime information
    about how long it will take to be picked up or about taxi's current position and code.
    In addition, MyTaxiService provides an efficient way to allocate taxies by dividing the
    city in zones and using a queue based allocation system, in order to reduce the
    waiting time and city's traffic.

    \subsection{Goals}
      In this subsection we describe a set of high level goals that myTaxiService is proposed to achieve.\newline
      \begin{enumerate}
        \item Simplify and speed up the process of taking a taxi
        \begin{itemize}
          \item When a user has entered his taxi ride details and clicks or taps
          the request button, then MyTaxiService will find the first
          available taxi that fits for the inserted ride details, booking it to the user

          \item When a user has entered his taxi reservation details and clicks or taps
          the book button, then MyTaxiService will book a taxi that fits for the
          inserted booking details and for the indicated meeting time
        \end{itemize}
        \item Guarantee an efficient and fair management of taxi queues
        \begin{itemize}
          \item Guarantee a right distribution of taxies in the city
          \item Guarantee short taxi availability times and short waiting times
         \end{itemize}
      \end{enumerate}
    \subsection{Glossary}
        \subsubsection{Terms disambiguation}
        \begin{description}
         	\item [MyTaxiService(F)] \hfill \\The front end of MyTaxiService, that is to say the components
          	of the application that manage the interaction with the user and the logic behind
          	this interaction
        	\item [MyTexiService(B)]\hfill \\ The back end of MyTaxiService, that is to say the components
          	of the application that manage the forwarding of the ride / reservation request with all
          	their associated notifications, the search of available taxies that are compatible with the request/reservation inserted, and other internal
          	tasks not exposed to the user or the taxies
        	\item  [MyTaxiService] \hfill \\MyTaxiService(F) + MyTaxiService(B)
        	\item [Taxi driver] \hfill \\The person who is licensed to drive a taxi cab.
        	\item [Taxi(with capital T)]\hfill \\A taxi that uses the MyTaxiService
        	\item [Request] \hfill \\An electronic message sent by a user through MyTaxiService(F)
          	to MyTaxiService(B). This electronic message refers to the case in which the user wants to be picked up by a taxi asap.
        	\item [Reservation]\hfill \\ An electronic message sent by a user through MyTaxiService(F)
           to MyTaxiService(B). This electronic message refers to the case in which the user wants to be picked up by a taxi at a specific time.
        	\item [Zone] \hfill \\An area of the city.
         	\item [Credentials]\hfill \\ A combination of username and password, used by a registered user to access the myTaxiService application.
        	\item [Taxi ride] \hfill \\A movement of people, through a taxi cab, from one geogrphical point to another
        	\item [Queue] \hfill \\A data structure managed with a FIFO(First in First Out) policy.
       	\item [User] \hfill \\A person that wants to take a taxi and is not registered to MyTaxiService.
       	\item [Registered User] \hfill \\A person that needs to take a taxi and is registered to MyTaxiService.
       \end{description}

      \subsubsection{Acronyms}
      \begin{description}
        \item [RASD:] Requirements Analysis and Specification Document
        \item [FIFO:] First In First Out
       \end{description}

    \subsection{Reference Documents}
       \begin{itemize}
       	\item Specification Document: MyTaxiService-AA2015-2016.pdf
        	\item IEEE Std 830-1998 IEEE Recommended Practice for Software Requirements Specifications.
        	\item IEEE Std 1016 tm -2009 Standard for Information Tecnology-System Design-Software Design Descriptions.
       \end{itemize}
    \subsection{Assumptions}
    \begin{enumerate}
        \item MyTaxiService has been commissioned by the city local government.
        Each taxi in the city must have a taxi license provided by the local government. This license requires the registration to the myTaxiService
        and forces drivers to accept ride request/reservations only using MyTaxiService.
        During the registration process taxi drivers are asked to provide their personal and vehicle data and work time table.
        At the end of the registration process a unique code will be assigned to each taxi.
        Eventually, the city's local government provides each taxi a device. This device is used to see incoming ride requests or reservations
        and to signal their acceptance

        \item Since MyTaxiService is aware of the work timetable of each taxi and GPS data, taxies are considered unavailable
        if and only if they are serving a ride/reservation request.  After a taxi has finished serving a passenger he has to notify that
        to myTaxiService and so myTaxiService will consider the taxi available again. Considering that myTaxiService knows the position
        of the taxi and the destination of each ride there is no chance a taxi drive can cheat by not signalling he hasn't yet finished service a passenger.

        \item A Taxi might have an accident, if that happens, the taxi driver can report it and so the taxi
        is considered unavailable.

        \item Once a user sends a ride request, he or she cannot change any detail of the request nor can
          undo the request.

        \item MyTaxiService is aware of the characteristics of each taxies (number of passengers)

        \item MyTaxiService is aware of all the possible valid location in the city, so the user is forced to select one of them and not insert one of them.

        \item Taxi can accept only ride reservation/request within the city borders.

        \item Accepting a ride/reservation request by the taxi driver: signaling that the taxi has already left in order to pick a registered user up.

        \item Visitor: A generic person that is not registered to the service.

        \item Exists an unique application(myTaxiService(F)) both for taxi driver and normal user, myTaxiService(F) give access to the proper set of functionalities
        according to the login credentials.
    \end{enumerate}

\section{Requirements Specification}
  \subsection{Main Actors}
    In this section are defined all the actors that interact with the myTaxiService during its operation.
    \begin{description}
      \item [Visitor] \hfill \\
          A user can only see the login page and complete the registration(which is mandatory to use the service)
          to be able to access to all the functionality of the application.

      \item [Registered User]\hfill \\
          A registered user can, after a successful login, access all the functionality of the application: it can
          request/reserve a taxi ride, view the status of older reservations and cancel a already confirmed taxi reservation.

      \item [Taxi driver] \hfill \\
          A taxi driver, after a successful login, is granted to: being notified by the system of an incoming request,
          notify the system about the conclusion of a taxi ride.

      \item [MyTaxiService(B)] \hfill \\
          This actor represents the back-end part of the myTaxiService software system as described in the Glossary
    \end{description}

    \subsection {Functional Requirements}


    \subsection {Non Functional Requirements}

    \subsection {Constraints}

    \subsection {Jackson-Zave approach}

    \subsection {Scenarios definition}

\section {UML Diagrams}
  \subsection Use Case diagram
\end{document}
