% !TEX TS-program = pdflatex
% !TEX encoding = UTF-8 Unicode

% This is a simple template for a LaTeX document using the "article" class.
% See "book", "report", "letter" for other types of document.

\documentclass[11pt,titlepage]{article} % use larger type; default would be 10pt

\usepackage[utf8]{inputenc} % set input encoding (not needed with XeLaTeX)

%%% Examples of Article customizations
% These packages are optional, depending whether you want the features they provide.
% See the LaTeX Companion or other references for full information.

%%% PAGE DIMENSIONS
\usepackage{geometry} % to change the page dimensions
\geometry{a4paper} % or letterpaper (US) or a5paper or....
% \geometry{margin=2in} % for example, change the margins to 2 inches all round
% \geometry{landscape} % set up the page for landscape
%   read geometry.pdf for detailed page layout information

\usepackage{graphicx} % support the \includegraphics command and options
\usepackage{titlepic}

% \usepackage[parfill]{parskip} % Activate to begin paragraphs with an empty line rather than an indent

%%% PACKAGES
\usepackage{booktabs} % for much better looking tables
\usepackage{array} % for better arrays (eg matrices) in maths
\usepackage{paralist} % very flexible & customisable lists (eg. enumerate/itemize, etc.)
\usepackage{verbatim} % adds environment for commenting out blocks of text & for better verbatim
\usepackage{subfig} % make it possible to include more than one captioned figure/table in a single float
% These packages are all incorporated in the memoir class to one degree or another...

%%% HEADERS & FOOTERS
\usepackage{fancyhdr} % This should be set AFTER setting up the page geometry
\pagestyle{empty} % options: empty , plain , fancy
\renewcommand{\headrulewidth}{0pt} % customise the layout...
\lhead{}\chead{}\rhead{}
\lfoot{}\cfoot{\thepage}\rfoot{}

%%% SECTION TITLE APPEARANCE
\usepackage{sectsty}
\allsectionsfont{\sffamily\mdseries\upshape} % (See the fntguide.pdf for font help)
% (This matches ConTeXt defaults)

%%% ToC (table of contents) APPEARANCE
\usepackage[nottoc,notlof,notlot]{tocbibind} % Put the bibliography in the ToC
\usepackage[titles,subfigure]{tocloft} % Alter the style of the Table of Contents
\renewcommand{\cftsecfont}{\rmfamily\mdseries\upshape}
\renewcommand{\cftsecpagefont}{\rmfamily\mdseries\upshape} % No bold!

%%% END Article customizations

%%% The "real" document content comes below...

\titlepic{\includegraphics[scale=0.60]{polimi_logo.jpg}}
\title{\textbf{D}esign \textbf{D}ocument \\ \vspace{1cm} \large{Version 1.0}} 
\author{Giorgio Pea(Mat. 853872), Andrea Sessa(Mat. 850082)}
\date{13/11/2015} 

\begin{document}

\maketitle

\newpage

\tableofcontents

\newpage

\section{Introduction}

\subsection{Purpose}
    This document represents the Design Document(DD).
    The purpose of the Design Document is to provide a medium/base level description of the design of MyTaxiService
    in order to allow for software developers to proceed with an understanding of what is to be
    built and how it is expected to be built.\newline
    The main goal of this document is to completely describe the system-to-be by:
    \begin{itemize}
      \item Detecting high-level components of the software to be
      \item Describing how these components communicate and interact with each other
      \item Describing how software components are distributed on the architecture's tiers
      \item Motivating and describing the adopted architectural style
   \end{itemize}

\subsection{Scope}
    The aim of this project is to develop MyTaxiService, a web/mobile application that
    makes easier and quicker taking taxies within the city’s borders. Thanks to MyTaxiService,
    anyone can request or book a taxi and get realtime information about how long
    it will take to be picked up or about the taxi’s current position and identification code.
    In addition to that, MyTaxiService provides an efficient way to allocate taxies by dividing
    the city in zones and using a queue based allocation system, in order to reduce the
    average waiting time and city’s traffic.\newline
    This Software Design is focused on the base level system and critical parts
    of the system.

\subsection{Terms Definition}
	\subsubsection{Glossary}
		\begin{itemize}
			\item \textbf{Tier:} Refers to a possible hardware level in a generic architecture
	        		\item \textbf{Layer:} Refers to a possible software level in a generic software system
			\item \textbf{Mtaxi:} A taxi that joined MyTaxiService
		\end{itemize}
	\subsubsection{Acronyms}
		\begin{itemize}
		        \item \textbf{DD:} Design Document
		        \item \textbf{MVC:} Model View Controller
		        \item \textbf{FIFO:} First In First Out
		        \item \textbf{API:} Application Programming Interface
		        \item \textbf{GUI:} Graphic User Interface
		        \item \textbf{GPS:} Global Positioning System
		\end{itemize}
		

\subsection{Reference Documents}
	\begin{itemize}
		\item RASD version 1.1
	\end{itemize}

\subsection{Document Structure}
	\begin{description}
	     \item [Introduction] \hfill \\
	      This section provides a general description of the Design Document by clearly stating purpose an aim of the project.
	      It also includes a disambiguation section to help the reader in the process of resolving the ambiguity generated by
	      the use of natural language
	
	     \item [Architecture Design] \hfill \\
	      The first part of this section provides a detailed description of the high-level components of MyTaxiService and of how
	      these components interact.
	      The second part introduces the architectural style chosen for MyTaxiService. The focus is on motivation, advantages and possible
	      disadvantages of the chosen architecture.
	
	    \item [Algorithms Design] \hfill \\
	      The section aims to provide a very medium/low level description of some routine functionalities of MyTaxiService.
	      Some code is included.
	
	    \item [User Interface Design] \hfill \\
	      In this section are provided some mockups describing the requirements of the user interface to-be
	
	    \item [Requirements Traceability] \hfill \\
	      This section provides a matrix of traceability that allows the reader to map functional requirements on the
	      previously defined software components
	\end{description}

\newpage

\section{Architectural Design}

\subsection{Overview}
         The architectural design section is divided into two main parts:
	\begin{description}
	        \item [Software components description and interaction] \hfill \\
	            In this section is provided a detailed description of the main components of the software system
	            and of their interactions.
	            To exemplify the above mentioned description a set of UML diagrams(Component, Deployment, Sequence diagrams) is
	            included.
	        \item [Architectural styles and patterns] \hfill \\
	            In this section the software system architecture is illustrated using a schematic diagram.
	            In addition to that all the architectural choices, patterns and styles considered are motivated and described.
	\end{description}

\subsection{High level components and their interaction}

\subsection{Component View}

\subsection{Deployment View}

\subsection{Runtime View}

\subsection{Components Interface}

\subsection{Selected architectural styles and patterns}	
 
\newpage
\section{Algorithms Design}

\subsection{Overview}

\newpage
\section{User Interface Design}
	No new useful user interface needs to be specified in this section.\newline
	For a detailed description of the user interfaces, please refer to section 2.2.2 of the RASD v1.1 document.

\newpage
\section{Requirements Traceability}

\end{document}
